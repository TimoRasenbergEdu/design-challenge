Playing games is a popular pastime for many people. Not only is the gaming
industry is a multi-billion dollar industry that has been growing rapidly over
the past few years, but rise of mobile gaming, more people are playing games
than ever before. In fact, according to statistics~\cite{gamer-count} there are
currently 3.32 billion gamers worldwide.

Besides being a popular pastime, games have also been used for educational
purposes. Games can be used to teach a wide range of subjects, from history to
mathematics, but even soft skills like teamwork and leadership.

However, a portion of the population may experience difficulties when playing
games which require social interaction with others. This can be due to a
variety of reasons, such as social anxiety, introversion, or other reasons for
not wanting to play with others.

Another issue experienced by some players is that they may not be very skilled
at the game they are playing, and may feel that they are holding their team
back. This can lead to feelings of frustration and anxiety, and can even cause
some players to stop playing the game altogether.

In this thesis, we will explore the use of \gls{ai}, and in particular
\gls{rl}, to create a system that can play games with human players in a way
that is both enjoyable and educational.