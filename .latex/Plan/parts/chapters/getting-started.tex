Thus far, we have discussed the general structure of the system, as well its
methods, data requirements, and ethical aspects. However, time constraints 
will not allow us to create a version of the system that can be applied to
any or even multiple games.

Therefore, we will initially focus on creating a system that can perform
as described on a single game. This game will be chosen based on its
feasibility, popularity, and potential for educational value. Some
considerations are, but are not limited to
\begin{itemize}
    \item Tetris-style games, such as \textit{Tetris}, \textit{Puyo Puyo},
    or \textit{Dr. Mario}. These games are simple to understand, but can be
    difficult to master, making them suitable for both casual and competitive
    play. This difficulty also makes these types of games suitable for applying
    our system with the purpose of providing guidance to players whom are
    inexperienced or want to improve. They also have potential for educational
    value, such as teaching problem-solving skills and strategy.
    \item Block stacking games, such as \textit{Jenga} or \textit{Build Up}.
    Like Tetris-style games, these games are simple to understand, but may not
    be as hard to master. However, this makes these type of games suitable for
    applying our system with the purpose of providing another player to play
    with or against. They also have potential for educational value, such as
    teaching spatial reasoning and physics.
\end{itemize}

Thus, the first step in creating the system will be to choose a game to focus
on, and to create a version of the game that can be used for training and
evaluation purposes. This means it should be possible to perform actions,
read the state, rewards, and detect terminated sessions. As mentioned in
\hyperref[sec:innovation-methods]{innovation methods}, the game will be created
using HTML, CSS, and JavaScript.

Once the game has been created, the next step will be to create a version of
the system that can play the game. This will involve creating agents that can
interact with the game, and training those agents using the \gls{neat}
algorithm. However, the \gls{neat} algorithm is complex, and will require
time and effort to understand and implement. Thus, we will count our first
step as a success if we can create the chosen game, and create a version of
the system that can interact with the game, even if the agents are not yet
trained.
