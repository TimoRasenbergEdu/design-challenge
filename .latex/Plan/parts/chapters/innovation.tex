As mentioned in the \hyperref[ch:intro]{introduction}, the main goal of this
thesis is to provide a system that can play games with human players in a way
that is both enjoyable and educational. This means that the system should be
able to play games against human players, play games with human players, or
even assist human players in playing games.

\section{Methods}\label{sec:innovation-methods}
The system will be built using a combination of existing technologies and
techniques, with a focus on \gls{rl}. \gls{rl} is a type of machine learning
that allows an agent to learn how to interact with an environment by
reinforcing good behavior and punishing bad behavior. This makes it well-suited
for learning how to play games, as games are essentially environments that can
be interacted with.

Furthermore, a personal requirement of this thesis, is that the system's 
agents should learn through the \gls{neat} algorithm. This algorithm is a type
of \gls{ne} that evolves neural networks by adding and removing nodes and
connections, which allows for the creation of complex neural networks that can
solve complex problems.

Lastly, the games that the system will play will be created using HTML, CSS,
and JavaScript, unless another library or framework is found that is better
suited for the task, or a suitable game can be found that can be used for
training and evaluation purposes.

\section{Data}\label{sec:innovation-data}
Due to the training methods used in this thesis, the system will not require
any data to be collected or stored. Instead, the system will generate its own
data by playing games (with human players), and using the results of those
games to improve its performance.

However, the system will require a way to store the neural networks that it
evolves, as well as any retrieved metrics, so that they can be reused in future
games, and utilized for future observations. This will be done by storing the
neural networks in a file format that can be easily loaded and used by the
system.

\section{Ethics}\label{sec:innovation-ethics}
The system will be designed with ethics in mind, and will be built in a way
that respects the privacy and autonomy of human players. This means that the
system will not collect any personal data from human players, and will not
store any data that could be used to identify them.

One of the potential ethical concerns of this thesis is the possibility of
discouraging human players who struggle with social interaction, the target
audience of this thesis, from learning to grow and develop their social skills.
However, on an educational level, where such growth takes place, it is
also the responsibility of teachers, parents, and guardians to ensure that the
system is used in a way that is beneficial to the human players.
