Gymnasium also provides another representation for the Q*bert environment,
which is the RAM state. The RAM state is a one-dimensional tensor with a size
of 128, which represents the memory of the Atari 2600 console. The RAM state
contains information about the game, such as the player's score, the number of
lives, and the position of the player. The RAM state is a smaller
representation of the environment compared to the grayscaled and \gls{rgb}
states, which makes it easier to use in experiments.

However, the state representation is not as intuitive as the grayscaled and
\gls{rgb} states, as it does not provide a visual representation of the
environment. This lack of visual representation may make it harder for the
agent to learn the environment, as it does not have a direct view of the
environment. Furthermore, the lack of visual representation also impairs the
user's ability to visually observe the agent's behaviour, as the user cannot
see the environment that the agent is interacting with. This also makes it
hard to demonstrate the agent's behaviour to others.

Despite these drawbacks, the RAM state is still a useful representation of the
environment, as it provides a more compact representation of the environment
compared to the grayscaled and \gls{rgb} states. Although this representation
will not be used in our experiments due to the reasons mentioned above, it is
still a valuable representation that can be used in other experiments.