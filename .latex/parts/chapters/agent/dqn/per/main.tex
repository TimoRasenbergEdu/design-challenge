\Gls{per} is an extension of the \gls{dqn} algorithm that prioritizes the
transitions based on their temporal-difference error. The transitions with
higher temporal-difference error are sampled more frequently, which helps to
improve the learning process. The \gls{per} algorithm uses a sum-tree data
structure to store the transitions and their priorities. The sum-tree data
structure allows for efficient sampling of the transitions based on their
priorities.

% \begin{algorithm}[H]
%     \begin{algorithmic}[1]
%         \State Initialize replay memory $D$ to capacity $N$
%         \State Initialize sum-tree $T$ with capacity $N$
%         \State Initialize action-value function $Q$ with random weights $\theta$
%         \For{episode $= 1, M$}
%             \State Initialize sequence $s_1 = \{x_1\}$ and preprocessed sequence $\phi_1 = \phi(s_1)$
%             \For{t $= 1, T$}
%                 \State With probability $\epsilon$ select a random action $a_t$
%                 \State otherwise select $a_t = \argmax_a Q(\phi(s_t), a; \theta)$
%                 \State Execute action $a_t$ in emulator and observe reward $r_t$ and image $x_{t+1}$
%                 \State Set $s_{t+1} = s_t, a_t, x_{t+1}$ and preprocess $\phi_{t+1} = \phi(s_{t+1})$
%                 \State Store transition $(\phi_t, a_t, r_t, \phi_{t+1})$ in $D$
%                 \State Calculate temporal-difference error $\delta_t = r_t + \gamma \max_{a^\prime} Q(\phi_{t+1}, a^\prime; \theta) - Q(\phi_t, a_t; \theta)$
%                 \State Update priority $p_t = |\delta_t| + \epsilon$
%                 \State Insert transition $(\phi_t, a_t, r_t, \phi_{t+1})$ with priority $p_t$ into $T$
%                 \State Sample random minibatch of transitions $(\phi_j, a_j, r_j, \phi_{j+1})$ from $T$
%                 \State Set $y_j = \begin{cases}
%                     r_j & \text{if episode terminates at step } j+1 \\
%                     r_j + \gamma \max_{a^\prime} Q(\phi_{j+1}, a^\prime; \theta) & \text{otherwise}
%                 \end{cases}$
%                 \State Perform a gradient descent step on $(y_j - Q(\phi_j, a_j; \theta))^2$ with respect to the network parameters $\theta$
%                 \State Update priority $p_j = |\delta_j| + \epsilon$
%                 \State Update priority of transition $(\phi_j, a_j, r_j, \phi_{j+1})$ in $T$ to $p_j$
%             \EndFor
%         \EndFor
%     \end{algorithmic}
%     \caption{Prioritized Experience Replay}
%     \label{alg:per}
% \end{algorithm}


\begin{algorithm}[H]
    \begin{algorithmic}[1]
        \State Initialize minibatch $k$, step-size $n$, replay period $K$, and
            size $N$, exponents $\alpha$ and $\beta$, and budget $T$.
        \State Initialize replay memory $H = \theta$, $\delta = 0$, $p_1 = 1$
        \State Observe $S_0$ and choose $A_0 \sim \pi_\theta(S_0)$
        \For{$t = 1$ to $T$ do}
            \State Observe $S_t$, $R_t$, $\gamma_t$
            \State store transition $(S_{t-1}, A_{t-1}, R_t, \gamma_t, S_t)$ in
                $H$ with maximal priority $p_t = \max_{i \leq t} p_i$
            \If{$t \% K = 0$}
                \For{$j = 1$ to $k$ do}
                    \State Sample transition $j \sim P(j) = p_j^\alpha / \sum_i p_i^\alpha$
                    \State Compute importance-sampling weight $w_j = (N \cdot P(j))^{-\beta} / \max_i w_i$
                    \State Compute TD-error $\delta_j = R_j + \gamma_j Q_{target}(S_j, \argmax_a Q(S_j, a)) - Q(S_{j-1}, A_{j-1})$
                    \State Update transition priority $p_j = |\delta_j|$
                    \State Accumulate weight-change $\delta \leftarrow \delta + w_j \cdot \delta_j \cdot \nabla_\theta Q(A_j | S_j)$
                \EndFor
                \State Update weights $\theta \leftarrow \theta + n \cdot \delta$, reset $\delta = 0$
                \State From time to time, copy weights into target network $\theta_{target} \leftarrow \theta$
            \EndIf
            \State Choose action $A_t \sim \pi_\theta(S_t)$
        \EndFor
    \end{algorithmic}
    \caption{Prioritized Experience Replay}
    \label{alg:per}
\end{algorithm}

However, the \gls{per} algorithm has a bias that can affect the learning
process. The bias is caused by the fact that the priorities are updated
frequently, which can lead to overestimation of the Q-values. To address this
issue, the \gls{per} algorithm uses importance sampling to correct the bias,
as seen on line 10 of \cref{alg:per}.

\subsubsection{Importance Sampling}\label{sec:agent-dqn-per-importance-sampling}
Importance sampling is a technique used to estimate the expected value of
a function under a different distribution. In the context of reinforcement
learning, importance sampling is used to estimate the expected value of the
Q-values under the target policy, given that the Q-values are estimated under
the behavior policy. This is useful when the target policy is different from
the behavior policy, as it allows the agent to learn how to act optimally under
the target policy.

In the context of the \gls{per} algorithm, importance sampling is used to
correct the bias introduced by the prioritized replay. The bias is caused by
the fact that the priorities are updated frequently, which can lead to
overestimation of the Q-values. To correct this bias, the \gls{per} algorithm
uses importance sampling to reweight the updates to the Q-values based on the
importance of the transitions. This helps to reduce the bias and improve the
stability of the learning process.

The importance sampling weight is calculated as follows:
\begin{equation}
    w_i = \left( \frac{1}{N} \cdot \frac{1}{P(i)} \right)^{\beta}
\end{equation}