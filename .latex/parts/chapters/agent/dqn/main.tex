\gls{dqn} is a reinforcement learning algorithm that combines Q-learning
with deep learning. The algorithm uses a neural network to approximate the
Q-values of the state-action pairs. The Q-values are updated using the Bellman
equation, which is defined as follows:
\begin{equation}
    Q(S_t, A_t) \leftarrow Q(S_t, A_t) + \alpha [R_{t+1} + \gamma \max_{a} Q(S_{t+1}, a) - Q(S_t, A_t)]
\end{equation}

The \hyperref[alg:dqn]{algorithm} uses experience replay to store and sample
transitions from the environment. The transitions are stored in a replay
memory, which is a fixed-size buffer. The algorithm samples a random minibatch
of transitions from the replay memory to update the Q-values. This helps to
break the correlation between the samples and stabilize the training process.

\begin{algorithm}[H]
    \begin{algorithmic}[1]
        \State Initialize replay memory $D$ to capacity $N$
        \State Initialize action-value function $Q$ with random weights $\theta$
        \For{episode $= 1, M$}
            \State Initialize sequence $s_1 = \{x_1\}$ and preprocessed sequence $\phi_1 = \phi(s_1)$
            \For{t $= 1, T$}
                \State With probability $\epsilon$ select a random action $a_t$
                \State otherwise select $a_t = \argmax_a Q(\phi(s_t), a; \theta)$
                \State Execute action $a_t$ in emulator and observe reward $r_t$ and image $x_{t+1}$
                \State Set $s_{t+1} = s_t, a_t, x_{t+1}$ and preprocess $\phi_{t+1} = \phi(s_{t+1})$
                \State Store transition $(\phi_t, a_t, r_t, \phi_{t+1})$ in $D$
                \State Sample random minibatch of transitions $(\phi_j, a_j, r_j, \phi_{j+1})$ from $D$
                \State Set $y_j = \begin{cases}
                    r_j & \text{if episode terminates at step } j+1 \\
                    r_j + \gamma \max_{a^\prime} Q(\phi_{j+1}, a^\prime; \theta) & \text{otherwise}
                \end{cases}$
                \State Perform a gradient descent step on $(y_j - Q(\phi_j, a_j; \theta))^2$ with respect to the network parameters $\theta$
            \EndFor
        \EndFor
    \end{algorithmic}
    \caption{Deep Q-learning}
    \label{alg:dqn}
\end{algorithm}

\subsection{Baseline}\label{sec:agent-dqn-baseline}
\textit{To be added.}

\subsection{Prioritized Experience Replay}\label{sec:agent-dqn-per}
\subsubsection{Importance Sampling}\label{sec:agent-dqn-per-importance-sampling}
Importance sampling is a technique used to estimate the expected value of
a function under a different distribution. In the context of reinforcement
learning, importance sampling is used to estimate the expected value of the
Q-values under the target policy, given that the Q-values are estimated under
the behavior policy. This is useful when the target policy is different from
the behavior policy, as it allows the agent to learn how to act optimally under
the target policy.

In the context of the \gls{per} algorithm, importance sampling is used to
correct the bias introduced by the prioritized replay. The bias is caused by
the fact that the priorities are updated frequently, which can lead to
overestimation of the Q-values. To correct this bias, the \gls{per} algorithm
uses importance sampling to reweight the updates to the Q-values based on the
importance of the transitions. This helps to reduce the bias and improve the
stability of the learning process.

The importance sampling weight is calculated as follows:
\begin{equation}
    w_i = \left( \frac{1}{N} \cdot \frac{1}{P(i)} \right)^{\beta}
\end{equation}

\subsection{Double Deep Q-learning}\label{sec:agent-dqn-ddqn}
Much like \gls{dqn}, \gls{ddqn} is a reinforcement learning algorithm that
combines Q-learning with deep learning. The algorithm uses a neural network to
approximate the Q-values of the state-action pairs, and also updates the
Q-values using the Bellman equation, which is defined as follows:
\begin{equation}
    Q(S_t, A_t) \leftarrow Q(S_t, A_t) + \alpha [R_{t+1} + \gamma \max_{a} \hat{Q}(S_{t+1}, a) - Q(S_t, A_t)]    
\end{equation}

The key difference between \gls{dqn} and \gls{ddqn} is that the Q-values are
updated using the maximum Q-value of the next state according to the target
action-value function. This helps to reduce the overestimation of the Q-values
and improve the stability of the training process. This target action-value
function is referred to as $\hat{Q}$, and is updated every $C$ steps to match
the action-value function $Q$. Like \gls{dqn} \hyperref[alg:ddqn]{algorithm}
uses experience replay.

\begin{algorithm}[H]
    \begin{algorithmic}[1]
        \State Initialize replay memory $D$ to capacity $N$
        \State Initialize action-value function $Q$ with random weights $\theta$
        \State Initialize target action-value function $\hat{Q}$ with weights $\theta^- = \theta$
        \For{episode $= 1, M$}
            \State Initialize sequence $s_1 = \{x_1\}$ and preprocessed sequence $\phi_1 = \phi(s_1)$
            \For{t $= 1, T$}
                \State With probability $\epsilon$ select a random action $a_t$
                \State otherwise select $a_t = \argmax_a Q(\phi(s_t), a; \theta)$
                \State Execute action $a_t$ in emulator and observe reward $r_t$ and image $x_{t+1}$
                \State Set $s_{t+1} = s_t, a_t, x_{t+1}$ and preprocess $\phi_{t+1} = \phi(s_{t+1})$
                \State Store transition $(\phi_t, a_t, r_t, \phi_{t+1})$ in $D$
                \State Sample random minibatch of transitions $(\phi_j, a_j, r_j, \phi_{j+1})$ from $D$
                \State Set $y_j = \begin{cases}
                    r_j & \text{if episode terminates at step } j+1 \\
                    r_j + \gamma \max_{a^\prime} \hat{Q}(\phi_{j+1}, a^\prime; \theta^-) & \text{otherwise}
                \end{cases}$
                \State Perform a gradient descent step on $(y_j - Q(\phi_j, a_j; \theta))^2$ with respect to the network parameters $\theta$
                \State Every $C$ steps reset $\hat{Q} = Q$
            \EndFor
        \EndFor
    \end{algorithmic}
    \caption{Double Deep Q-learning}
    \label{alg:ddqn}
\end{algorithm}