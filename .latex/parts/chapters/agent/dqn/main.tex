\gls{dqn} is a reinforcement learning algorithm that combines Q-learning
with deep learning. The algorithm uses a neural network to approximate the
Q-values of the state-action pairs. The Q-values are updated using the Bellman
equation, which is defined as follows:
\begin{equation}
    Q(S_t, A_t) \leftarrow Q(S_t, A_t) + \alpha [R_{t+1} + \gamma \max_{a} Q(S_{t+1}, a) - Q(S_t, A_t)]
\end{equation}

The \hyperref[alg:dqn]{algorithm} uses experience replay to store and sample
transitions from the environment. The transitions are stored in a replay
memory, which is a fixed-size buffer. The algorithm samples a random minibatch
of transitions from the replay memory to update the Q-values. This helps to
break the correlation between the samples and stabilize the training process.

\begin{algorithm}[H]
    \begin{algorithmic}[1]
        \State Initialize replay memory $D$ to capacity $N$
        \State Initialize action-value function $Q$ with random weights $\theta$
        \For{episode $= 1, M$}
            \State Initialize sequence $s_1 = \{x_1\}$ and preprocessed sequence $\phi_1 = \phi(s_1)$
            \For{t $= 1, T$}
                \State With probability $\epsilon$ select a random action $a_t$
                \State otherwise select $a_t = \argmax_a Q(\phi(s_t), a; \theta)$
                \State Execute action $a_t$ in emulator and observe reward $r_t$ and image $x_{t+1}$
                \State Set $s_{t+1} = s_t, a_t, x_{t+1}$ and preprocess $\phi_{t+1} = \phi(s_{t+1})$
                \State Store transition $(\phi_t, a_t, r_t, \phi_{t+1})$ in $D$
                \State Sample random minibatch of transitions $(\phi_j, a_j, r_j, \phi_{j+1})$ from $D$
                \State Set $y_j = \begin{cases}
                    r_j & \text{if episode terminates at step } j+1 \\
                    r_j + \gamma \max_{a^\prime} Q(\phi_{j+1}, a^\prime; \theta) & \text{otherwise}
                \end{cases}$
                \State Perform a gradient descent step on $(y_j - Q(\phi_j, a_j; \theta))^2$ with respect to the network parameters $\theta$
            \EndFor
        \EndFor
    \end{algorithmic}
    \caption{Deep Q-learning}
    \label{alg:dqn}
\end{algorithm}

\subsection{Baseline}\label{sec:agent-dqn-baseline}
\input{parts/chapters/agent/dqn/baseline.tex}

\subsection{Prioritized Experience Replay}\label{sec:agent-dqn-per}
\section{State Representation}\label{sec:environment-state-representation}
\section{State Representation}\label{sec:environment-state-representation}
\section{State Representation}\label{sec:environment-state-representation}
\input{parts/chapters/environment/state-representation/main.tex}

\section{Reward Function}\label{sec:environment-reward-function}
\input{parts/chapters/environment/reward-function.tex}

\section{Reward Function}\label{sec:environment-reward-function}
The Q*bert environment contains a reward function that provides the agent
with feedback on its actions. The reward function is a function that assigns a
reward to the agent based on the state of the environment and the action taken
by the agent.

The reward function for the Q*bert environment is defined as follows:
\begin{enumerate}
    \item Q*bert changes cube to destination color: 25 points
    \item Q*bert catches Sam: 300 points
    \item Q*bert catches green ball: 100 points
    \item Q*bert lures Coily off pyramid: 500 points
    \item Bonus points for every round you complete: 3100 points
\end{enumerate}

As we can see, the rewards in the Q*bert environment are large, which may make
it easier for the agent to differentiate between good and bad actions. However,
the large rewards may also make it harder for the agent to learn the
environment, as it may take longer for the agent to converge to an optimal
policy. To address this issue, we will use reward clipping, which will clip the
rewards to a range of -1 to 1. This clipping will reduce the scale of the
rewards, which will make it easier for the agent to learn the environment.

\section{Reward Function}\label{sec:environment-reward-function}
The Q*bert environment contains a reward function that provides the agent
with feedback on its actions. The reward function is a function that assigns a
reward to the agent based on the state of the environment and the action taken
by the agent.

The reward function for the Q*bert environment is defined as follows:
\begin{enumerate}
    \item Q*bert changes cube to destination color: 25 points
    \item Q*bert catches Sam: 300 points
    \item Q*bert catches green ball: 100 points
    \item Q*bert lures Coily off pyramid: 500 points
    \item Bonus points for every round you complete: 3100 points
\end{enumerate}

As we can see, the rewards in the Q*bert environment are large, which may make
it easier for the agent to differentiate between good and bad actions. However,
the large rewards may also make it harder for the agent to learn the
environment, as it may take longer for the agent to converge to an optimal
policy. To address this issue, we will use reward clipping, which will clip the
rewards to a range of -1 to 1. This clipping will reduce the scale of the
rewards, which will make it easier for the agent to learn the environment.

\subsection{Double Deep Q-learning}\label{sec:agent-dqn-ddqn}
Much like \gls{dqn}, \gls{ddqn} is a reinforcement learning algorithm that
combines Q-learning with deep learning. The algorithm uses a neural network to
approximate the Q-values of the state-action pairs, and also updates the
Q-values using the Bellman equation, which is defined as follows:
\begin{equation}
    Q(S_t, A_t) \leftarrow Q(S_t, A_t) + \alpha [R_{t+1} + \gamma \max_{a} \hat{Q}(S_{t+1}, a) - Q(S_t, A_t)]    
\end{equation}

The key difference between \gls{dqn} and \gls{ddqn} is that the Q-values are
updated using the maximum Q-value of the next state according to the target
action-value function. This helps to reduce the overestimation of the Q-values
and improve the stability of the training process. This target action-value
function is referred to as $\hat{Q}$, and is updated every $C$ steps to match
the action-value function $Q$. Like \gls{dqn} \hyperref[alg:ddqn]{algorithm}
uses experience replay.

\begin{algorithm}[H]
    \begin{algorithmic}[1]
        \State Initialize replay memory $D$ to capacity $N$
        \State Initialize action-value function $Q$ with random weights $\theta$
        \State Initialize target action-value function $\hat{Q}$ with weights $\theta^- = \theta$
        \For{episode $= 1, M$}
            \State Initialize sequence $s_1 = \{x_1\}$ and preprocessed sequence $\phi_1 = \phi(s_1)$
            \For{t $= 1, T$}
                \State With probability $\epsilon$ select a random action $a_t$
                \State otherwise select $a_t = \argmax_a Q(\phi(s_t), a; \theta)$
                \State Execute action $a_t$ in emulator and observe reward $r_t$ and image $x_{t+1}$
                \State Set $s_{t+1} = s_t, a_t, x_{t+1}$ and preprocess $\phi_{t+1} = \phi(s_{t+1})$
                \State Store transition $(\phi_t, a_t, r_t, \phi_{t+1})$ in $D$
                \State Sample random minibatch of transitions $(\phi_j, a_j, r_j, \phi_{j+1})$ from $D$
                \State Set $y_j = \begin{cases}
                    r_j & \text{if episode terminates at step } j+1 \\
                    r_j + \gamma \max_{a^\prime} \hat{Q}(\phi_{j+1}, a^\prime; \theta^-) & \text{otherwise}
                \end{cases}$
                \State Perform a gradient descent step on $(y_j - Q(\phi_j, a_j; \theta))^2$ with respect to the network parameters $\theta$
                \State Every $C$ steps reset $\hat{Q} = Q$
            \EndFor
        \EndFor
    \end{algorithmic}
    \caption{Double Deep Q-learning}
    \label{alg:ddqn}
\end{algorithm}